\documentclass[12pt]{article}
\usepackage[english,russian]{babel}
\usepackage{amsmath,amssymb,amsthm,latexsym,amsfonts}
\usepackage[utf8]{inputenc}
\usepackage[english,russian]{babel}
\begin{document}
1. Если функция непрерывна на отрезке ${\displaystyle [a,b]}$ и на его концах принимает значения разных знаков, то внутри отрезка найдется хотя бы одна точка, в которой функция обращается в нуль .\\

2. Если функция ${\displaystyle f}$ непрерывна на сегменте и на своих концах принимает значение разных знаков, то существует такая точка ${\displaystyle c}$, принадлежащая этому отрезку, в которой ${\displaystyle f(c)=0}$ .\\

3. Пусть ${\displaystyle f}$ непрерывна на отрезке ${\displaystyle [a,b]}$ и на концах этого отрезка принимает значения разных знаков, тогда найдется точка ${\displaystyle \xi}$ на интервале ${\displaystyle (a,b)}$, в которой значение функции равно нулю .\\

4. Если непрерывная функция, определённая на вещественном интервале, принимает два значения, то она принимает и любое значение между ними .\\

5. Пусть функция ${\displaystyle f}$ непрерывна на отрезке ${\displaystyle [a,b]}$ , причем ${\displaystyle f(a) \\neq f(b)}$ , тогда для любого числа ${\displaystyle С}$, заключенного между ${\displaystyle f(a)}$ и ${\displaystyle f(b)}$ , найдется точка ${\displaystyle \gamma \in (a,b)}$, что ${\displaystyle f(\gamma)=C}$ .\\

6. Если функция непрерывна на отрезке ${\displaystyle [a,b]}$, то, принимая любые два значения на ${\displaystyle [a,b]}$, функция принимает и всякое промежуточное значение .\\

7. Пусть функция ${\displaystyle f}$ непрерывна на отрезке ${\displaystyle [a,b]}$ и ${\displaystyle f(a)<f(b)}$ , то для любого числа ${\displaystyle A}$ такого, что ${\displaystyle f(a)<A<f(b)}$, найдется точка ${\displaystyle с}$ из интервала ${\displaystyle (a,b)}$, в которой ${\displaystyle f(c)=A}$ .\\

8. Пусть функция ${\displaystyle f}$ непрерывна на отрезке ${\displaystyle [a,b]}$ и пусть ${\displaystyle С}$ есть произвольное число, находящееся между значениями ${\displaystyle f(a)}$ и ${\displaystyle f(b)}$ , тогда существует точка ${\displaystyle с \in [a,b]}$, для которой ${\displaystyle f(с)=С}$ .\\

9. Если функция непрерывна на отрезке  ${\displaystyle [a,b]}$, то среди ее значений на этом отрезке имеется наименьшее и наибольшее значение .\\

10. Если функция ${\displaystyle f}$ непрерывна на отрезке ${\displaystyle [a,b]}$, то она ограничена на нём и притом достигает своих минимального и максимального значений, т.е. существуют ${\displaystyle x_{m},\;x_{M}\in [a,b]}$ такие, что ${\displaystyle f(x_{m})\leq f(x)\leq f(x_{M})}$ для всех ${\displaystyle x\in [a,b]}$.\\

11. Если ${\displaystyle f}$ непрерывна на ${\displaystyle [a,b]}$, то она достигает на нем своей верхней и нижней грани.\\

12. Если функция ${\displaystyle f(x)}$ монотонна (нестрого) на отрезке ${\displaystyle [a,b]}$ , а функция ${\displaystyle g(x)}$ интегрируема на ${\displaystyle [a,b]}$ , то существует точка  ${\displaystyle \xi \in [a;b]}$ такая , что ${\displaystyle \int \limits _{a}^{b} f(x)\cdot g(x)\, dx=f(a)\cdot \int \limits _{a}^{\xi} g(x)\, dx+f(b)\cdot \int \limits _{\xi}^{b} g(x)\, dx }$ .\\

13. Если ${\displaystyle f,g \in R_{[a,b]}}$ и функция ${\displaystyle f(x)}$ монотонна на ${\displaystyle [a,b]}$, то найдется такая точка ${\displaystyle \xi \in [a;b]}$ , такие что ${\displaystyle \xi \in [a,b]}$ такая , что ${\displaystyle \int \limits _{a}^{b} f(x)\cdot g(x)\, dx=f(a)\cdot \int \limits _{a}^{\xi} g(x)\, dx+f(b)\cdot \int \limits _{\xi}^{b} g(x)\, dx }$ .\\

14. Если в промежутке  ${\displaystyle [a,b]}$ функции ${\displaystyle u(x)}$ и  ${\displaystyle v(x)}$ непрерывны и имеют непрерывные производные, то ${\displaystyle \int _{a}^{b} u(x)dv(x)=(u(x)\cdot v(x)){\vert}^{b}_{a}=\int \limits _{a}^{b} v(x)\, du(x)}$ .\\

15. Пусть функции ${\displaystyle u}$ и ${\displaystyle v}$ дифференцируемы на некотором интервале и пусть функция ${\displaystyle u^{\prime}(x)*v(x)}$ имеет первообразную на этом интервале, тогда функция ${\displaystyle u(x)*v^{\prime}(x)}$ также имеет первообразную на этом интервале, причем справедливо равенство ${\displaystyle \int \limits u(x)*v^{\prime}(x) \, dx=u(x)*v(x)-\int \limits v(x)*u^{\prime}(x)\, dx}$ .\\

16. Если функция ${\displaystyle f(x)}$ кусочно-непрерывна на промежутке ${\displaystyle [a,b]}$, то на этом промежутке она интегрируема, т.е. существует ${\displaystyle \int _{a}^{b} f(x) \, dx}$ .\\

17. Если функция кусочно-непрерывна на некотором отрезке, то она интегрируема на этом отрезке .\\

18. Для того, чтобы ограниченная функция ${\displaystyle f}$ была интегрируема в смысле Дарбу, необходимо и достаточно, чтобы для любого ${\displaystyle \varepsilon>0}$ нашлось разбиение ${\displaystyle P_{\varepsilon}}$ такое, что ${\displaystyle S(f, P_{\varepsilon})-s(f, P_{\varepsilon})<\varepsilon}$ .\\

19. Пусть функция ${\displaystyle f}$ ограничена на отрезке ${\displaystyle \left[ {a,b} \right]}$, тогда для того, чтобы ${\displaystyle f}$ была интегрируемой на этом отрезке, необходимо и достаточно, чтобы  для любого положительного ${\displaystyle \varepsilon}$ существовало такое положительное ${\displaystyle \delta}$, что для каждого разбиения ${\displaystyle \Pi}$ , диаметр которого ${\displaystyle d\left( \Pi \right) < \delta}$, справедливо неравенство ${\displaystyle {\overline S_{\Pi} } - {\underline S_{\Pi}} < \varepsilon}$.\\

20. Ограниченная функция ${\displaystyle f}$ интегрируема по Риману на отрезке ${\displaystyle [a,b]}$ тогда и только тогда, когда для любого ${\displaystyle \varepsilon>0}$ можно найти такое разбиение ${\displaystyle P}$ указанного отрезка, что выполняется неравенство: ${\displaystyle \sum \limits _{k=1}^{n} (M_k-m_k)\Delta x_k<\varepsilon}$, где ${\displaystyle m_k=inf \{f(t):t\in[x_{k-1},x_k]\}}$ и ${\displaystyle M_k=sup \{f(t):t\in[x_{k-1},x_k]\}}$ .\\

21. Ограниченная функция ${\displaystyle f}$  интегрируема по Риману на отрезке ${\displaystyle [a,b]}$ тогда и только тогда, когда для любого положительного числа ${\displaystyle \varepsilon}$ существует такое разбиение ${\displaystyle P\in \sigma[a,b]}$, что выполняется неравенство ${\displaystyle U(P,f) - L(P,f) < 0}$ .\\

22. Если функция ${\displaystyle f(x)}$ дифференцируема в некоторой точке  ${\displaystyle x_0}$, принадлежащей интервалу ${\displaystyle (x_0-\delta,x_0+\delta)}$ и имеет в этой точке экстремум, то ${\displaystyle f^{\prime}(x_0)=0}$ .\\

23. Если функция ${\displaystyle y=f(x)}$ имеет экстремум в точке ${\displaystyle x_0}$, то ее производная ${\displaystyle f^{\prime}(x_0)}$ либо равна нулю, либо не существует.\\

24. Пусть ${\displaystyle \varphi \geq 0}$ - суммируемая на ${\displaystyle A}$ функция и ${\displaystyle с>0}$ - произвольное положительное число, тогда ${\displaystyle \mu\{x \in A \vert \varphi(x) \geq c\} \leq \dfrac{1}{c}\int \limits _{A} \varphi(x) \, d\mu}$ .\\

25. Если функция ${\displaystyle f \in L_1(X)}$, то для любого ${\displaystyle \varepsilon>0}$ верно неравенство ${\displaystyle \mu\{|f| \geq \varepsilon\} \leq \dfrac{1}{\varepsilon}*\int \limits _{X} |f| \, d\mu}$ .\\

26. Если ${\displaystyle f(x)}$ непрерывна на промежутке ${\displaystyle [a,b]}$, то имеет место такая оценка определенного интеграла: ${\displaystyle m(b-a)\leq \int _{a}^{b} f(x)\,dx\leq M(b-a)}$, где ${\displaystyle m}$ - наименьшее, а ${\displaystyle M}$ - наибольшее значения функции ${\displaystyle f(x)}$ на промежутке ${\displaystyle [a,b]}$ .\\

27. Если ${\displaystyle m}$ есть наименьшее, а ${\displaystyle M}$ - наибольшее значение функции ${\displaystyle f(x)}$ в промежутке ${\displaystyle (a,b)}$, то значение интеграла ${\displaystyle \int \limits _{a}^{b} f(x) \, dx}$ заключено между ${\displaystyle m(b-a)}$ и ${\displaystyle M(b-a)}$ .\\

28. Если функция непрерывна на отрезке  ${\displaystyle [a,b]}$, то на этом отрезке она и ограничена .\\

29. Если ${\displaystyle f}$ непрерывна на ${\displaystyle [a,b]}$, то она ограничена на нем, т.е. существует такое число ${\displaystyle M}$, что ${\displaystyle |f(x)| \leq M}$, при всех ${\displaystyle x \in [a,b]}$ .\\

30. Пусть функция ${\displaystyle f(x)}$ непрерывна на отрезке  ${\displaystyle [a,b]}$ и дифференцируема на интервале ${\displaystyle (a,b)}$ , тогда для того, чтобы ${\displaystyle f(x)}$ была постоянна на отрезке ${\displaystyle [a,b]}$, необходимо и достаточно, чтобы для любого ${\displaystyle x \in (a,b)}$ выполнялось условие ${\displaystyle f^{\prime}=0}$ .\\

31. Если во всех точках некоторого промежутка производная функции ${\displaystyle f(x)}$ равна нулю, то функция ${\displaystyle f(x)}$ сохраняет в этом промежутке постоянное значение .\\

32. Пусть функция ${\displaystyle f(x)}$ непрерывна на отрезке  ${\displaystyle [a,b]}$ и дифференцируема на интервале ${\displaystyle (a,b)}$ ,  тогда чтобы ${\displaystyle f(x)}$ не убывала на отрезке ${\displaystyle [a,b]}$, необходимо и достаточно, чтобы при любом ${\displaystyle x \in (a,b)}$ выполнялось условие ${\displaystyle f^{\prime} \geq 0}$.\\

33. Пусть функция ${\displaystyle f(x)}$ непрерывна на отрезке  ${\displaystyle [a,b]}$ и дифференцируема на интервале ${\displaystyle (a,b)}$ ,  тогда чтобы ${\displaystyle f(x)}$ не возрастала на отрезке ${\displaystyle [a,b]}$, необходимо и достаточно, чтобы при любом ${\displaystyle x \in (a,b)}$ выполнялось условие ${\displaystyle f^{\prime} \leq 0}$.\\

34. Если во всех точках некоторого промежутка производная функции f больше нуля, то функция f возрастает в этом промежутке .\\

35. Если во всех точках некоторого промежутка производная функции g меньше нуля, то функция g убывает на этом промежутке.\\

36. Если ${\displaystyle P(z)}$ - полином степени ${\displaystyle n \geq 1}$ и для любого комплексного числа ${\displaystyle с}$ существует полином ${\displaystyle Q(z)}$ степени ${\displaystyle n-1}$ такой, что справедливо равенство ${\displaystyle P(z)=(z-c)\cdot Q(z)+P(c)}$ .\\

37. Остаток при делении многочлена ${\displaystyle P(z)}$ на многочлен ${\displaystyle z-a}$ равен значению этого многочлена при ${\displaystyle z=a}$ , то есть равен ${\displaystyle P(a)}$ .\\

38. Пусть ${\displaystyle f}$ — непрерывная функция, определённая на отрезке ${\displaystyle [a,b]}$ , тогда для любого ${\displaystyle \varepsilon >0}$ существует такой многочлен ${\displaystyle p}$  с вещественными коэффициентами, что для всех ${\displaystyle x}$  из ${\displaystyle [a,\;b]}$ одновременно выполнено условие ${\displaystyle |f(x)-p(x)|<\varepsilon }$ .\\

39. Если функция ${\displaystyle f(x)}$ непрерывна на сегменте ${\displaystyle [a,b]}$, то для ${\displaystyle \varepsilon>0}$ найдется многочлен ${\displaystyle P_n(x)}$ с номером ${\displaystyle n}$, зависящим от ${\displaystyle \varepsilon}$, такой, что ${\displaystyle |P_n(x)-f(x)|<\varepsilon}$ сразу для всех ${\displaystyle x}$ из сегмента ${\displaystyle [a,b]}$ .\\

40. Если на отрезке ${\displaystyle [a,b]}$ функции ${\displaystyle f(x)}$ и ${\displaystyle g(x)}$ непрерывны и дифференцируемы в каждой точке интервала ${\displaystyle (a,b)}$, причем  ${\displaystyle g^{\prime}\neq 0}$ во всех точках этого интервала, то тогда между точками ${\displaystyle a}$ и  ${\displaystyle b}$ существует точка ${\displaystyle c}$ ${\displaystyle (a<c<b)}$, что имеет место равенство ${\displaystyle \dfrac{f(b)-f(a)}{g(b)-g(a)}=\dfrac{f^{\prime}(c)}{g^{\prime}(c)}}$ .\\

41. Если функции ${\displaystyle f(x)}$ и ${\displaystyle g(x)}$ определены и непрерывны на сегменте ${\displaystyle [a,b]}$ , ${\displaystyle f(x)}$ и ${\displaystyle g(x)}$ имеют конечные производные ${\displaystyle f^{\prime}(x)}$  и ${\displaystyle g^{\prime}(x)}$ на интервале ${\displaystyle (a,b)}$, ${\displaystyle {f^{\prime}}^2(x)+{g^{\prime}}^2(x) \neq 0}$ при ${\displaystyle a<x<b}$, ${\displaystyle g(a)\neq g(b)}$, то ${\displaystyle \dfrac{f(b)-f(a)}{g(b)-g(a)}=\dfrac{f^{\prime}(c)}{g^{\prime}(c)}}$, где ${\displaystyle a<c<b}$.\\

42. Если функция ${\displaystyle y=f(x)}$ непрерывна на отрезке ${\displaystyle [a,b]}$ и дифференцируема на интервале ${\displaystyle (a,b)}$, то внутри отрезка ${\displaystyle [a,b]}$ найдется хотя бы одна точка ${\displaystyle с}$ ${(\displaystyle a<c<b)}$ такая, что будет иметь равенство ${\displaystyle f(b)-f(a)=f^{\prime}(c)*(b-a)}$ .\\

43. Если функция ${\displaystyle f(x)}$ определена и непрерывна на сегменте ${\displaystyle [a,b]}$ и ${\displaystyle f(x)}$ имеет конечную производную ${\displaystyle f^{\prime}(x)}$ на интервале ${\displaystyle (a,b)}$, то ${\displaystyle f(b)-f(a)=(b-a)\cdot f^{\prime}(c)}$, где ${\displaystyle a<c<b}$ .\\

44. Если функция ${\displaystyle f(x)}$ и ${\displaystyle g(x)}$ дифференцируемы в окрестности точки ${\displaystyle x_0}$ и, кроме того, ${\displaystyle \lim _{x \to x_0} f(x)=0}$ и ${\displaystyle \lim _{x \to x_0} g(x)=0}$ причем ${\displaystyle g^{\prime}\neq 0}$ в окрестности точки ${\displaystyle x_0}$, то тогда ${\displaystyle \lim _{x \to x_0} \dfrac{f(x)}{g(x)}=\lim _{x \to x_0} \dfrac{f^{\prime}(x)}{g^{\prime}(x)}}$ при условии, что второй предел существует .\\

45. Пусть функции ${\displaystyle f(x)}$ и ${\displaystyle g(x)}$ определены и дифференцируемы в промежутке ${\displaystyle (a,b)}$, ${\displaystyle g^{\prime}(x) \neq 0}$ для всех ${\displaystyle x\in (a,b)}$, ${\displaystyle \lim \limits _{x \to a} f(x)=0}$ и ${\displaystyle \lim \limits _{x \to a} g(x)=0}$  существует конечный или бесконечный предел ${\displaystyle \lim \limits _{x \to a} \dfrac{f^{\prime}(x)}{g^{\prime}(x)}}$, тогда ${\displaystyle \lim \limits _{x \to a}\dfrac{f(x)}{g(x)}=\lim \limits _{x \to a} \dfrac{f^{\prime}(x)}{g^{\prime}(x)}}$ .\\

46. Если ${\displaystyle f(x)}$ непрерывна на промежутке ${\displaystyle [a,b]}$, то между точками ${\displaystyle a}$ и ${\displaystyle b}$ найдется хотя бы одна точка  ${\displaystyle \xi}$ такая, что будет иметь место равенство ${\displaystyle \int \limits _{a}^{b} f(x) dx=f(\xi)\cdot (b-a)}$ .\\

47. Пусть функция ${\displaystyle f(x)}$ непрерывна на отрезке ${\displaystyle [a,b]}$, тогда существует точка ${\displaystyle \xi \in [a,b]}$ такая, что ${\displaystyle \int \limits _{a}^{b} f(x) dx=f(\xi)\cdot (b-a)}$ .\\

48. Если вещественная функция, непрерывная на отрезке ${\displaystyle [a,b]}$ и дифференцируемая на интервале ${\displaystyle (a,b)}$ , принимает на концах отрезка ${\displaystyle [a,b]}$ одинаковые значения, то на интервале ${\displaystyle (a,b)}$ найдётся хотя бы одна точка, в которой производная функции равна нулю.\\

49. Пусть функция  дифференцируема в открытом промежутке , на концах этого промежутка сохраняет непрерывность и принимает одинаковые значения: ${\displaystyle f(a)=f(b)}$ , тогда существует точка  , в которой производная функции  равна нулю : ${\displaystyle f'(c)=0}$  .\\

50. Пусть функция ${\displaystyle f(x)}$ непрерывна на отрезке ${\displaystyle [a,b]}$ и дифференцируема на интервале ${\displaystyle (a,b)}$, причем ${\displaystyle f(a)=f(b)}$ , тогда существует точка ${\displaystyle c \in [a, b]}$ такая, что ${\displaystyle f'(c)=0}$ .\\

51. Если функция f непрерывна на ${\displaystyle [a,b]}$, функция f дифференцируема во всех внутренних точках ${\displaystyle [a,b]}$ и ${\displaystyle f(a)=f(b)}$, тогда существует точка ${\displaystyle c\in (a,b)}$, в которой ${\displaystyle f^{\prime}(c)=0}$ .\\

52. Если функция ${\displaystyle f(x)}$  непрерывна на отрезке ${\displaystyle [a,b]}$, в каждой точке интервала ${\displaystyle (a,b)}$ существует конечная производная ${\displaystyle f^{\prime}(x)}$ и, кроме того,${\displaystyle f(a)=f(b)}$, то тогда между точками ${\displaystyle a}$ и ${\displaystyle b}$ найдется хотя бы одна точка ${\displaystyle с}$ ${\displaystyle a<c<b}$ такая, что ${\displaystyle f^{\prime}(c)=0}$ .\\

53. Если функция ${\displaystyle f}$ интегрируема на ${\displaystyle [a,b]}$ и непрерывна в точке ${\displaystyle x_0 \in [a,b]}$, то функция ${\displaystyle F(x)=\int \limits _{a}^{x} f(t)\,dt}$  дифференцируема в точке функция ${\displaystyle x_0}$ и функция ${\displaystyle F^{\prime}(x_0)=f(x_0)}$ .\\

54. Если функция ${\displaystyle f(x)}$ непрерывна на промежутке ${\displaystyle [a,b]}$, то интеграл с переменным верхним пределом  ${\displaystyle \int \limits _{a}^{x} f(t)\,dt}$ имеет производную, равную значению подынтегральной функции при верхнем пределе, т.е ${\displaystyle \left(\int \limits _{a}^{x} f(t)\,dt\right)^{\prime}=f(x)}$ .\\

55. Если функция ${\displaystyle f}$ интегрируема на ${\displaystyle [a,b]}$ и непрерывна в точке ${\displaystyle с \in [a,b]}$, то интеграл с переменным верхним пределом ${\displaystyle F(x)=\int \limits _{a}^{x} f(t)\, dt}$ имеет производную в точке ${\displaystyle с}$ и выполняется равенство ${\displaystyle F^{\prime}(c)=f(c)}$.\\

56. Если ${\displaystyle \textstyle f(x)}$ непрерывна на отрезке ${\displaystyle \left[a,b\right]}$  и ${\displaystyle \textstyle \Phi (x)}$ — любая её первообразная на этом отрезке, то имеет место равенство ${\displaystyle \int \limits _a^b f(x)\,dx=\Phi (b)-\Phi (a)={\Bigg .}\Phi (x){\Bigg |}_{a}^{b}}$ .\\

57. Если  ${\displaystyle F (x)}$ – любая первообразная функции ${\displaystyle f (x)}$,  то справедливо равенство ${\displaystyle \int \limits _a^x f(t)\,dt=F(x)-F(a)}$ .\\

58. Пусть ${\displaystyle f}$ - функция, интегрируема по Риману на отрезке ${\displaystyle [a,b]}$ и функция ${\displaystyle F}$ непрерывна на отрезке [a,b] и дифференцируема в каждой внутренней точке этого отрезка, причем ${\displaystyle F^{\prime}=f(x)}$ , ${\displaystyle a<x<b}$, тогда справедлива формула ${\displaystyle \int \limits _{a}^{b} f(x) \, dx=F(b)-F(a)}$ .\\

59. Если функция ${\displaystyle y=f(x)}$ определенна на отрезке ${\displaystyle [a,b]}$ и на этом отрезке ${\displaystyle f(x)}$ дифференцируема ${\displaystyle n}$ раз, тогда ${\displaystyle f(x)}$ может быть представлена в виде ${\displaystyle f(x)=f(a)+\dfrac{f^{\prime}(a)}{1!}*(x-a)+\dfrac{f^{''}(a)}{2!}*(x-a)^2+\ldots+\dfrac{f^{(n-1)}(a)}{(n-1)!}*(x-a)^{n-1}+R_n}$ .\\

60. Пусть ${\displaystyle k \geq 1}$ является целым, и пусть функция ${\displaystyle f : \mathbb{R} \rightarrow \mathbb{R}}$ является ${\displaystyle k}$ раз дифференцируемой в точке ${\displaystyle a \in \mathbb{R}}$, тогда существует функция ${\displaystyle h_k: \mathbb{R} \rightarrow \mathbb{R}}$ такая, что ${\displaystyle f(x)=f(a)+\dfrac{f^{\prime}(a)}{1!}(x-a)+\dfrac{f^{''}(a)}{2!}(x-a)^2+\ldots+\dfrac{f^{(k)}(a)}{k!}*(x-a)^k+h_k*(x-a)^k}$, ${\displaystyle \lim \limits _{x \to a} h_k(a)=0}$ .\\

61. Пусть функция ${\displaystyle f(x)}$ ${\displaystyle n}$ раз дифференцируема в точке ${\displaystyle x_0}$, тогда ${\displaystyle f(x)=\sum \limits _{k=0}^{n} \dfrac{f^{(k)}(x_0)}{k!}*(x-x_0)^k+o((x-x_0)^n)}$ .\\

\end{document}