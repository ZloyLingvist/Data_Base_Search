\documentclass[12pt]{article}
\usepackage[english,russian]{babel}
\usepackage{amsmath,amssymb,amsthm,latexsym,amsfonts}
\usepackage[utf8]{inputenc}
\usepackage[english,russian]{babel}
\begin{document}
1. Если функция непрерывна на отрезке ${\displaystyle [a,b]}$ и на его концах принимает значения разных знаков, то внутри отрезка найдется хотя бы одна точка, в которой функция обращается в нуль .\\

2. Если функция ${\displaystyle f}$ непрерывна на сегменте и на своих концах принимает значение разных знаков, то существует такая точка ${\displaystyle c}$, принадлежащая этому отрезку, в которой ${\displaystyle f(c)=0}$ .\\

3. Пусть ${\displaystyle f}$ непрерывна на отрезке ${\displaystyle [a,b]}$ и на концах этого отрезка принимает значения разных знаков, тогда найдется точка ${\displaystyle \xi}$ на интервале ${\displaystyle (a,b)}$, в которой значение функции равно нулю .\\

4. Если непрерывная функция, определённая на вещественном интервале, принимает два значения, то она принимает и любое значение между ними .\\

5. Пусть функция ${\displaystyle f}$ непрерывна на отрезке ${\displaystyle [a,b]}$ , причем ${\displaystyle f(a) \\neq f(b)}$ , тогда для любого числа ${\displaystyle С}$, заключенного между ${\displaystyle f(a)}$ и ${\displaystyle f(b)}$ , найдется точка ${\displaystyle \gamma \in (a,b)}$, что ${\displaystyle f(\gamma)=C}$ .\\

6. Если функция непрерывна на отрезке ${\displaystyle [a,b]}$, то, принимая любые два значения на ${\displaystyle [a,b]}$, функция принимает и всякое промежуточное значение .\\

7. Пусть функция ${\displaystyle f}$ непрерывна на отрезке ${\displaystyle [a,b]}$ и ${\displaystyle f(a)<f(b)}$ , то для любого числа ${\displaystyle A}$ такого, что ${\displaystyle f(a)<A<f(b)}$, найдется точка ${\displaystyle с}$ из интервала ${\displaystyle (a,b)}$, в которой ${\displaystyle f(c)=A}$ .\\

8. Пусть функция ${\displaystyle f}$ непрерывна на отрезке ${\displaystyle [a,b]}$ и пусть ${\displaystyle С}$ есть произвольное число, находящееся между значениями ${\displaystyle f(a)}$ и ${\displaystyle f(b)}$ , тогда существует точка ${\displaystyle с \in [a,b]}$, для которой ${\displaystyle f(с)=С}$ .\\

9. Пусть ${\displaystyle F}$ абсолютно непрерывная функция на ${\displaystyle [a,b]}$, тогда она почти всюду дифференцируема, обобщенная производная ${\displaystyle F^{\prime}}$ интегрируема по Лебегу и для всех ${\displaystyle x \in [a,b]}$ выполняется равенство : ${\displaystyle \int \limits _{a}^{x} F^{\prime}(t)\cdot dt=F(x)-F(a)}$ .\\

10. Если  ${\displaystyle f}$ абсолютно непрерывна на ${\displaystyle [a,b]}$, то ее производная существует почти всюду, интегрируема по Лебегу и удовлетворяет равенству: ${\displaystyle f(x)=f(a)+\int \limits _{a}^{x} f^{\prime}(t)*dt}$ для всех ${\displaystyle x \in [a,b]}$ .\\

11. Если функция непрерывна на отрезке  ${\displaystyle [a,b]}$, то среди ее значений на этом отрезке имеется наименьшее и наибольшее значение .\\

12. Если функция ${\displaystyle f}$ непрерывна на отрезке ${\displaystyle [a,b]}$, то она ограничена на нём и притом достигает своих минимального и максимального значений, т.е. существуют ${\displaystyle x_{m},\;x_{M}\in [a,b]}$ такие, что ${\displaystyle f(x_{m})\leq f(x)\leq f(x_{M})}$ для всех ${\displaystyle x\in [a,b]}$.\\

13. Если ${\displaystyle f}$ непрерывна на ${\displaystyle [a,b]}$, то она достигает на нем своей верхней и нижней грани.\\

14. Если функция ${\displaystyle f(x)}$ монотонна (нестрого) на отрезке ${\displaystyle [a,b]}$ , а функция ${\displaystyle g(x)}$ интегрируема на ${\displaystyle [a,b]}$ , то существует точка  ${\displaystyle \xi \in [a,b]}$ такая , что ${\displaystyle \int \limits _{a}^{b} f(x)\cdot g(x)\,dx=f(a)\cdot \int \limits _{a}^{\xi} g(x)\,dx+f(b)\cdot \int \limits _{\xi}^{b} g(x)\,dx }$ .\\

15. Если ${\displaystyle f,g \in R_{[a,b]}}$ и функция ${\displaystyle f(x)}$ монотонна на ${\displaystyle [a,b]}$, то найдется такая точка ${\displaystyle \xi \in [a,b]}$ , такие что ${\displaystyle \xi \in [a,b]}$ такая , что ${\displaystyle \int \limits _{a}^{b} f(x)\cdot g(x)\,dx=f(a)\cdot \int \limits _{a}^{\xi} g(x)\,dx+f(b)\cdot \int \limits _{\xi}^{b} g(x)\,dx }$ .\\

16. Если в промежутке  ${\displaystyle [a,b]}$ функции ${\displaystyle u(x)}$ и  ${\displaystyle v(x)}$ непрерывны и имеют непрерывные производные, то ${\displaystyle \int _{a}^{b} u(x)dv(x)=(u(x)\cdot v(x)){\vert}^{b}_{a}=\int \limits _{a}^{b} v(x)\,du(x)}$ .\\

17. Пусть функции ${\displaystyle u}$ и ${\displaystyle v}$ дифференцируемы на некотором интервале и пусть функция ${\displaystyle u^{\prime}(x)*v(x)}$ имеет первообразную на этом интервале, тогда функция ${\displaystyle u(x)*v^{\prime}(x)}$ также имеет первообразную на этом интервале, причем справедливо равенство ${\displaystyle \int \limits u(x)*v^{\prime}(x)*dx=u(x)*v(x)-\int \limits v(x)*u^{\prime}(x)*dx}$ .\\

18. Если функция ${\displaystyle f(x)}$ кусочно-непрерывна на промежутке ${\displaystyle [a,b]}$, то на этом промежутке она интегрируема, т.е. существует ${\displaystyle \int _{a}^{b} f(x)\,dx}$ .\\

19. Если функция кусочно-непрерывна на некотором отрезке, то она интегрируема на этом отрезке .\\

20. Если функция ${\displaystyle \varphi(x)\colon M \to E}$ непрерывна в некоторой точке ${\displaystyle a \in M}$, а функция ${\displaystyle f \colon E \to \mathbb{R}}$ непрерывна в соответствующей точке ${\displaystyle b=\varphi(a) \in E}$, то сложная функция ${\displaystyle F(t)=f(\varphi(t))}$ непрерывна в точке ${\displaystyle t=a}$ .\\

21. Пусть функция ${\displaystyle g(x)}$  непрерывна в точке ${\displaystyle x_0}$ и функция ${\displaystyle f(t)}$  непрерывна в точке ${\displaystyle t0 = g(x_0)}$, тогда сложная функция  ${\displaystyle f(g(x))}$ непрерывна в точке ${\displaystyle x_0}$ .\\

22. Для того, чтобы ограниченная функция ${\displaystyle f}$ была интегрируема в смысле Дарбу, необходимо и достаточно, чтобы для любого ${\displaystyle \varepsilon>0}$ нашлось разбиение ${\displaystyle P_{\varepsilon}}$ такое, что ${\displaystyle S(f, P_{\varepsilon})-s(f, P_{\varepsilon})<\varepsilon}$ .\\

23. Пусть функция ${\displaystyle f}$ ограничена на отрезке ${\displaystyle \left [ {a,b} \right ]}$, тогда для того, чтобы ${\displaystyle f}$ была интегрируемой на этом отрезке, необходимо и достаточно, чтобы  для любого положительного ${\displaystyle \varepsilon}$ существовало такое положительное ${\displaystyle \delta}$, что для каждого разбиения ${\displaystyle \Pi}$ , диаметр которого ${\displaystyle d\left ( \Pi \right ) < \delta}$, справедливо неравенство ${\displaystyle {\overline {S_{\Pi}} } - {\underline {S_{\Pi}}} < \varepsilon}$.\\

24. Ограниченная функция ${\displaystyle f}$ интегрируема по Риману на отрезке ${\displaystyle [a,b]}$ тогда и только тогда, когда для любого ${\displaystyle \varepsilon>0}$ можно найти такое разбиение ${\displaystyle P}$ указанного отрезка, что выполняется неравенство: ${\displaystyle \sum \limits _{k=1}^{n} (M_k-m_k)\Delta x_k<\varepsilon}$, где ${\displaystyle m_k=inf \{f(t):t\in[x_{k-1},x_k]\}}$ и ${\displaystyle M_k=sup \{f(t):t\in[x_{k-1},x_k]\}}$ .\\

25. Ограниченная функция ${\displaystyle f}$  интегрируема по Риману на отрезке ${\displaystyle [a,b]}$ тогда и только тогда, когда для любого положительного числа ${\displaystyle \varepsilon}$ существует такое разбиение ${\displaystyle P\in \sigma[a,b]}$, что выполняется неравенство ${\displaystyle U(P,f) - L(P,f) < \varepsilon}$ .\\

26. Если непрерывная функция ${\displaystyle f}$ на открытом интервале ${\displaystyle (a,b)}$ удовлетворяет равенству ${\displaystyle \int \limits _{a}^{b} f(x)\cdot h(x)\cdot dx=0}$ для всех финитных гладких функций ${\displaystyle h}$ на ${\displaystyle (a,b)}$, тогда ${\displaystyle f}$ является тождественным нулем .\\

27. Если функция  ${\displaystyle f(x)}$,  ${\displaystyle a \leq x \leq b}$, непрерывна и для любой непрерывной функции  ${\displaystyle h(x)}$, ${\displaystyle a \leq x \leq b}$, то  ${\displaystyle f(x) \equiv 0}$ .\\

28. Если ряд ${\displaystyle \sum \limits _{n=1}^{\infty} a_n}$ сходится, то его ${\displaystyle n}$-й член стремится к нулю при ${\displaystyle n \to \infty}$ .\\

29. Если ряд ${\displaystyle \sum \limits _{k=1}^{\infty} a_k}$ сходится, то ряд ${\displaystyle \lim \limits _{n \to \infty} a_n=0}$ сходится .\\

30. Если функция ${\displaystyle f(x)}$ дифференцируема в некоторой точке  ${\displaystyle x_0}$, принадлежащей интервалу ${\displaystyle (x_0-\delta,x_0+\delta)}$ и имеет в этой точке экстремум, то ${\displaystyle f^{\prime}(x_0)=0}$ .\\

31. Если функция ${\displaystyle y=f(x)}$ имеет экстремум в точке ${\displaystyle x_0}$, то ее производная ${\displaystyle f^{\prime}(x_0)}$ либо равна нулю, либо не существует.\\

32. Пусть функции ${\displaystyle f}$ и ${\displaystyle g}$ непрерывны в точке ${\displaystyle x_0}$, тогда функции ${\displaystyle f(x) \pm g(x)}$, ${\displaystyle f(x) \cdot g(x)}$ непрерывны в точке ${\displaystyle x_0}$ и если, дополнительно, ${\displaystyle g(x_0)\neq 0}$, то функция ${\displaystyle f(x)/g(x)}$ непрерывна в точке ${\displaystyle x_0}$ .\\

33. Если функции ${\displaystyle f(x)}$ и ${\displaystyle g(x)}$ определены в одном и том же промежутке ${\displaystyle X}$ и обе непрерывны в точке ${\displaystyle x_0}$, то в той же точке будут непрерывны и функции ${\displaystyle f(x) \pm g(x)}$,${\displaystyle f(x) \cdot g(x)}$, ${\displaystyle \dfrac{f(x)}{g(x)}}$ при условии, что ${\displaystyle g(x_0) \neq 0}$ .\\

34. Если функция ${\displaystyle f(x)}$ и ${\displaystyle g(x)}$ интегрируемы на отрезке ${\displaystyle [a,b]}$, ${\displaystyle p>1}$ , ${\displaystyle \dfrac{1}{p}+\dfrac{1}{q}=1}$, то справедливо неравенство ${\displaystyle |\int \limits _{a}^{b} f(x)\cdot g(x)\, dx| \leq \int \limits _{a}^{b} |f(x)\cdot g(x)| \, dx \leq (\int \limits _{a}^{b} |f(x)|^{p} \, dx)^{\dfrac{1}{p}}\cdot (\int \limits _{a}^{b} |g(x)|^{q} \, dx)^{\dfrac{1}{q}}}$ .\\

35. Пусть ${\displaystyle p,q \in (1,\infty)}$ , ${\displaystyle \dfrac{1}{p}+\dfrac{1}{q}=1}$, функция ${\displaystyle f}$ и ${\displaystyle g}$ интегрируемы на ${\displaystyle [a,b]}$ , тогда ${\displaystyle |\int \limits _{a}^{b} f(x)\cdot g(x)\, dx| \leq \int \limits _{a}^{b} |f(x)\cdot g(x)| \, dx \leq (\int \limits _{a}^{b} |f(x)|^{p} \, dx)^{\dfrac{1}{p}}\cdot (\int \limits _{a}^{b} |g(x)|^{q} \, dx)^{\dfrac{1}{q}}}$ .\\

36. Пусть на отрезке ${\displaystyle [a,b]}$ функции ${\displaystyle f(x)}$ и ${\displaystyle \alpha(x)}$ интегрируемы, ${\displaystyle m \leq f(x) \leq M}$ , ${\alpha(x) \geq 0}$ , ${\displaystyle \int \limits _{a}^{b} \alpha(x) dx=1}$, а функция ${\displaystyle \varphi}$ выпукла и непрерывна на отрезке ${\displaystyle [m,M]}$ , тогда имеет место неравенство ${\displaystyle \varphi \cdot (\int \limits _{a}^{b} f(x)\cdot \alpha(x)\,dx) \leq \int \limits _{a}^{b} \varphi(f(x))\cdot \alpha(x)\cdot dx}$ .\\

37. Для выпуклой функции ${\displaystyle \varphi(x)}$ и интегрируемой функции ${\displaystyle f(x)}$ выполняется неравенство ${\displaystyle \varphi\cdot (\int \limits _{a}^{b} f(x)\,dx) \leq \dfrac{1}{b-a}\cdot \int \limits _{a}^{b} \varphi((b-a)\cdot f(x))\,dx}$ .\\

38. Если функции ${\displaystyle f(x)}$ и ${\displaystyle g(x)}$ на отрезке ${\displaystyle [a,b]}$ ограничены и возрастают, то справедливо неравенство ${\displaystyle \int \limits _{a}^{b} f(x)\,dx \cdot \int \limits _{a}^{b} g(x)\,dx \leq (b-a)\cdot \int \limits _{a}^{b} f(x)\cdot g(x)\,dx}$ .\\

39. Пусть ${\displaystyle f}$ и ${\displaystyle g}$ заданы на ${\displaystyle [a,b]}$, причем ${\displaystyle f}$ возрастает, а ${\displaystyle g}$ убывает на ${\displaystyle [a,b]}$ , тогда ${\displaystyle \int \limits _{a}^{b} f(x) \cdot g(x)\,dx \leq \dfrac{1}{b-a}\cdot \int \limits _{a}^{b} f(x)\cdot \int \limits _{a}^{b} g(x)\,dx}$ .\\

40. Пусть ${\displaystyle \varphi \geq 0}$ - суммируемая на ${\displaystyle A}$ функция и ${\displaystyle с>0}$ - произвольное положительное число, тогда ${\displaystyle \mu\{x \in A \vert \varphi(x) \geq c\} \leq \dfrac{1}{c} \cdot \int \limits _{A} \varphi(x)d\mu}$ .\\

41. Если функция ${\displaystyle f \in L_1(X)}$, то для любого ${\displaystyle \varepsilon>0}$ верно неравенство ${\displaystyle \mu\{|f| \geq \varepsilon\} \leq \dfrac{1}{\varepsilon}\cdot \int \limits _{X} |f|d\mu}$ .\\

42. Если ${\displaystyle f(x)}$ непрерывна на промежутке ${\displaystyle [a,b]}$, то имеет место такая оценка определенного интеграла: ${\displaystyle m(b-a)\leq \int _{a}^{b} f(x)\,dx\leq M(b-a)}$, где ${\displaystyle m}$ - наименьшее, а ${\displaystyle M}$ - наибольшее значения функции ${\displaystyle f(x)}$ на промежутке ${\displaystyle [a,b]}$ .\\

43. Если ${\displaystyle m}$ есть наименьшее, а ${\displaystyle M}$ - наибольшее значение функции ${\displaystyle f(x)}$ в промежутке ${\displaystyle (a,b)}$, то значение интеграла ${\displaystyle \int \limits _{a}^{b} f(x)\,dx}$ заключено между ${\displaystyle m(b-a)}$ и ${\displaystyle M(b-a)}$ .\\

44. Если функция непрерывна на отрезке  ${\displaystyle [a,b]}$, то на этом отрезке она и ограничена .\\

45. Если ${\displaystyle f}$ непрерывна на ${\displaystyle [a,b]}$, то она ограничена на нем, т.е. существует такое число ${\displaystyle M}$, что ${\displaystyle |f(x)| \leq M}$, при всех ${\displaystyle x \in [a,b]}$ .\\

46. Если функция ${\displaystyle f(x)}$ имеет на интервале ${\displaystyle (a,b)}$ первообразную ${\displaystyle F(x)}$, то ${\displaystyle f(x)*dx = F(x) + C }$, где C — произвольная постоянная функция .\\

47. Если ${\displaystyle F(x)}$ есть первообразная для ${\displaystyle f(x)}$ на промежутке, тогда ${\displaystyle f(x)*dx = F(x) + C }$, где ${\displaystyle C}$ – произвольная постоянная.\\

48. Пусть функция ${\displaystyle g}$ непрерывно дифференцируема на отрезке ${\displaystyle [a,b]}$ и множеством ее значений является отрезок ${\displaystyle [a,b]}$, причем ${\displaystyle g(a)=c}$, ${\displaystyle g(b)=d}$, тогда если функция ${\displaystyle f}$ непрерывна на отрезке ${\displaystyle [c,d]}$, то выполняется равенство ${\displaystyle \int \limits _{c}^{d} f(x)\,dx=\int \limits _{a}^{b} f(g(t))\cdot g^{\prime}(t)\cdot dt}$ .\\

49. Пусть функция ${\displaystyle y = f(x)}$ непрерывна на отрезке ${\displaystyle [a,b]}$, функция ${\displaystyle x = g(t)}$ и ее производная ${\displaystyle g^{\prime}(t)}$ непрерывны при ${\displaystyle t \in [\alpha,beta]}$, множеством значений функции ${\displaystyle x = g(t)}$ при ${\displaystyle t \in [\alpha,beta]}$ является отрезок ${\displaystyle [a,b]}$ и ${\displaystyle g(a) = a}$, ${\displaystyle g(b) = b}$, то справедлива формула ${\displaystyle \int \limits _{a}^{b} f(x)\,dx=\int \limits _{\alpha}^{\beta} f(g(t))\cdot g^{\prime}(t)\cdot dt}$ .\\

50. Если ${\displaystyle f_n \in R[a,b], n \in \mathbb{N}}$, и последовательность ${\displaystyle \{f_n\}}$ сходится к функции ${\displaystyle f}$ равномерно на ${\displaystyle [a,b]}$, то ${\displaystyle f \in R[a,b]}$ и верно равенство ${\displaystyle \lim \limits _{n \to \infty} \int \limits _{a}^{b} f_n(x)\,dx=\int \limits _{a}^{b} f(x)\,dx}$ .\\

51. Пусть функции ${\displaystyle f_n}$ интегрируемы по Риману на отрезке ${\displaystyle [a,b]}$ и последовательность ${\displaystyle \{f_n\}}$ сходится к функции ${\displaystyle f}$ равномерно на этом отрезке, то функция ${\displaystyle f}$  интегрируема на отрезке ${\displaystyle [a,b]}$ и выполняется равенство ${\displaystyle \lim \limits _{n \to \infty} \int \limits _{a}^{b} f_n(x)\,dx=\int \limits _{a}^{b} f(x)\,dx}$ .\\

52. Пусть при всех ${k \geq n_0}$ и всех ${x \in X}$ выполняется неравенство ${|u_k(x)| \leq a_k}$ и числовой ряд ${\sum _{k=1}^{\infty} a_k}$ сходится, тогда функциональный ряд ${\sum _{k=1}^{\infty} u_k}$ сходится равномерно и абсолютно на множестве ${\displaystyle X}$ .\\

53. Пусть ${\sum _{{n=1}}^{{\infty }} u_{n}(x)}$ и существует последовательность ${\displaystyle a_{n}}$ такая, что для любого ${\displaystyle x\in X}$  выполняется неравенство ${\displaystyle |u_{n}(x)|<a_{n}}$, кроме того, ряд ${\displaystyle \sum _{n=1}^{\infty } a_{n}}$ сходится, тогда ряд ${\displaystyle \sum _{n=1}^{\infty } u_{n}(x)}$ сходится на множестве ${\displaystyle X}$ абсолютно и равномерно .\\

54. Пусть функция ${\displaystyle f(x)}$ непрерывна на отрезке  ${\displaystyle [a,b]}$ и дифференцируема на интервале ${\displaystyle (a,b)}$ , тогда для того, чтобы ${\displaystyle f(x)}$ была постоянна на отрезке ${\displaystyle [a,b]}$, необходимо и достаточно, чтобы для любого ${\displaystyle x \in (a,b)}$ выполнялось условие ${\displaystyle f^{\prime}=0}$ .\\

55. Если во всех точках некоторого промежутка производная функции ${\displaystyle f(x)}$ равна нулю, то функция ${\displaystyle f(x)}$ сохраняет в этом промежутке постоянное значение .\\

56. Пусть функция ${\displaystyle f(x)}$ непрерывна на отрезке  ${\displaystyle [a,b]}$ и дифференцируема на интервале ${\displaystyle (a,b)}$ ,  тогда чтобы ${\displaystyle f(x)}$ не убывала на отрезке ${\displaystyle [a,b]}$, необходимо и достаточно, чтобы при любом ${\displaystyle x \in (a,b)}$ выполнялось условие ${\displaystyle f^{\prime} \geq 0}$.\\

57. Если во всех точках некоторого промежутка производная функции ${\displaystyle f}$ больше нуля, то функция ${\displaystyle f}$ возрастает в этом промежутке .\\

58. Пусть функция ${\displaystyle f(x)}$ непрерывна на отрезке  ${\displaystyle [a,b]}$ и дифференцируема на интервале ${\displaystyle (a,b)}$ ,  тогда чтобы ${\displaystyle f(x)}$ не возрастала на отрезке ${\displaystyle [a,b]}$, необходимо и достаточно, чтобы при любом ${\displaystyle x \in (a,b)}$ выполнялось условие ${\displaystyle f^{\prime} \leq 0}$.\\

59. Если во всех точках некоторого промежутка производная функции ${\displaystyle g}$ меньше нуля, то функция ${\displaystyle g}$ убывает на этом промежутке.\\

60. Пусть функция ${\displaystyle f}$  непрерывна на промежутке ${\displaystyle [a,b]}$ и принимает на его концах значения разных знаков, тогда на этом промежутке существует такая точка  ${\displaystyle c}$, в которой ${\displaystyle f(c)=0}$ .\\

61. Если функция ${\displaystyle y = f(x)}$ непрерывна на отрезке ${\displaystyle [a,b]}$ и на концах этого отрезка принимает значения разных знаков, тогда внутри отрезка ${\displaystyle [a,b]}$ найдется, по крайней мере, одна точка ${\displaystyle x = C}$, в которой функция обращается в ноль: ${\displaystyle f(C) = 0}$, где ${\displaystyle a < C< b}$ .\\

62. Если функция ${\displaystyle f(x)}$ непрерывна на сегменте ${\displaystyle [a,b]}$ , то она и равномерно непрерывна на этом сегменте .\\

63. Если ${\displaystyle f}$ определена и непрерывна на отрезке ${\displaystyle [a,b]}$ , то она равномерно непрерывна на нем.\\

64. Если функция ${\displaystyle f}$ интегрируема по Риману на отрезке ${\displaystyle [a,b]}$, то функция ${\displaystyle f}$ интегрируема по Лебегу на этом отрезке и ${\displaystyle \int \limits _{[a,b]} f*d\mu=\int \limits _{a}^{b} f(x) dx}$ .\\

65. Если для функции, заданной на ${\displaystyle [a,b]}$, существует собственный интеграл Римана ${\displaystyle \int \limits _{a}^{b} f(x)d\mu}$, то она интегрируема и по Лебегу и ее интеграл Лебега ${\displaystyle \int \limits _{[a,b]} f(x)*d\mu}$ равен интегралу Римана .\\

66. Если ${\displaystyle P(z)}$ - полином степени ${\displaystyle n \geq 1}$ и для любого комплексного числа ${\displaystyle с}$ существует полином ${\displaystyle Q(z)}$ степени ${\displaystyle n-1}$ такой, что справедливо равенство ${\displaystyle P(z)=(z-c)\cdot Q(z)+P(c)}$ .\\

67. Остаток при делении многочлена ${\displaystyle P(z)}$ на многочлен ${\displaystyle z-a}$ равен значению этого многочлена при ${\displaystyle z=a}$ , то есть равен ${\displaystyle P(a)}$ .\\

68. Пусть ${\displaystyle f}$ — непрерывная функция, определённая на отрезке ${\displaystyle [a,b]}$ , тогда для любого ${\displaystyle \varepsilon >0}$ существует такой многочлен ${\displaystyle p}$  с вещественными коэффициентами, что для всех ${\displaystyle x}$  из ${\displaystyle [a,b]}$ одновременно выполнено условие ${\displaystyle |f(x)-p(x)|<\varepsilon }$ .\\

69. Если функция ${\displaystyle f(x)}$ непрерывна на сегменте ${\displaystyle [a,b]}$, то для ${\displaystyle \varepsilon>0}$ найдется многочлен ${\displaystyle P_n(x)}$ с номером ${\displaystyle n}$, зависящим от ${\displaystyle \varepsilon}$, такой, что ${\displaystyle |P_n(x)-f(x)|<\varepsilon}$ сразу для всех ${\displaystyle x}$ из сегмента ${\displaystyle [a,b]}$ .\\

70. Если функция ${\displaystyle f(x)}$ имеет конечную производную в промежутке ${\displaystyle [a,b]}$, то функция ${\displaystyle f^{\prime}(x)}$ принимает, в качестве значения, каждое промежуточное число между ${\displaystyle f^{\prime}(a)}$ и ${\displaystyle f^{\prime}(b)}$ .\\

71. Пусть функция ${\displaystyle f(x)}$ имеет производную на сегменте ${\displaystyle [a,b]}$, тогда для любого числа ${\displaystyle С}$, заключенного между ${\displaystyle A=f^{\prime}(a+0)}$ и ${\displaystyle B=f^{\prime}(b-0)}$, на этом сегменте найдется точка ${\displaystyle \xi}$ такая, что ${\displaystyle f^{\prime}(\xi)=C}$ .\\

72. Пусть ${\displaystyle E \subset \mathbb{R}^d}$ - произвольное множество, тогда для любой точки ${\displaystyle x \in conv}$ найдутся набор из ${\displaystyle d+1}$ точки ${\displaystyle x_1,\ldots,x_{d+1} \in E}$ и числа ${\displaystyle \lambda \geq 0}$, ${\sum \limits _{i=1}^{d+1} \lambda_i=1}$, такие, что ${\displaystyle  x=\sum \limits _{i=1}^{d+1} \lambda_i\cdot x_i}$ .\\

73. Пусть ${\displaystyle A\subset R^{m}}$  — компакт в ${\displaystyle m}$-мерном евклидовом пространстве, тогда ${\displaystyle \forall x\in coA}$ является выпуклой комбинацией не более чем ${\displaystyle m + 1}$ точек множества ${\displaystyle A}$ : ${\displaystyle coA=\left\{x:x=\sum _{i=1}^{m+1}\lambda _{i}x_{i}(x),\quad x_{i}(x)\in A,\quad \lambda _{i}\geqslant 0,\quad \sum _{i=1}^{m+1}\lambda _{i}=1,\quad i=1,\ 2,\ \dots ,\ m+1\right\}}$ .\\

74. Если на отрезке ${\displaystyle [a,b]}$ функции ${\displaystyle f(x)}$ и ${\displaystyle g(x)}$ непрерывны и дифференцируемы в каждой точке интервала ${\displaystyle (a,b)}$, причем  ${\displaystyle g^{\prime}\neq 0}$ во всех точках этого интервала, то тогда между точками ${\displaystyle a}$ и  ${\displaystyle b}$ существует точка ${\displaystyle c}$ ${\displaystyle (a<c<b)}$, что имеет место равенство ${\displaystyle \dfrac{f(b)-f(a)}{g(b)-g(a)}=\dfrac{f^{\prime}(c)}{g^{\prime}(c)}}$ .\\

75. Если функции ${\displaystyle f(x)}$ и ${\displaystyle g(x)}$ определены и непрерывны на сегменте ${\displaystyle [a,b]}$ , ${\displaystyle f(x)}$ и ${\displaystyle g(x)}$ имеют конечные производные ${\displaystyle f^{\prime}(x)}$  и ${\displaystyle g^{\prime}(x)}$ на интервале ${\displaystyle (a,b)}$, ${\displaystyle {f^{\prime}}^2(x)+{g^{\prime}}^2(x) \neq 0}$ при ${\displaystyle a<x<b}$, ${\displaystyle g(a)\neq g(b)}$, то ${\displaystyle \dfrac{f(b)-f(a)}{g(b)-g(a)}=\dfrac{f^{\prime}(c)}{g^{\prime}(c)}}$, где ${\displaystyle a<c<b}$.\\

76. Если функция ${\displaystyle y=f(x)}$ непрерывна на отрезке ${\displaystyle [a,b]}$ и дифференцируема на интервале ${\displaystyle (a,b)}$, то внутри отрезка ${\displaystyle [a,b]}$ найдется хотя бы одна точка ${\displaystyle с}$ ${(\displaystyle a<c<b)}$ такая, что будет иметь равенство ${\displaystyle f(b)-f(a)=f^{\prime}(c)*(b-a)}$ .\\

77. Если функция ${\displaystyle f(x)}$ определена и непрерывна на сегменте ${\displaystyle [a,b]}$ и ${\displaystyle f(x)}$ имеет конечную производную ${\displaystyle f^{\prime}(x)}$ на интервале ${\displaystyle (a,b)}$, то ${\displaystyle f(b)-f(a)=(b-a)\cdot f^{\prime}(c)}$, где ${\displaystyle a<c<b}$ .\\

78. Если в знакочередующемся ряде ${\displaystyle U_1-U_2+U_3-\ldots}$ члены таковы, что ${\displaystyle U_1>U_2>U_3>\ldots>0}$ и ${\displaystyle \lim \limits _{n \to \infty} U_n=0}$ , то знакочередующийся ряд сходится, его сумма положительна и не превосходит первого члена .\\

79. Если для знакочередующегося числового ряда ${\displaystyle u_1-u_2+u_3+\ldots+(-1)^{n-1}\cdot u_n+\ldots}$ выполнены условия ${\displaystyle u_1>u_2>\ldots>u_n>ldots}$ и ${\displaystyle \lim \limits _{n \to \infty} u_n=0}$, то ряд ${\displaystyle u_1-u_2+u_3+\ldots+(-1)^{n-1}\cdot u_n+\ldots}$ сходится, при этом сумма положительна и не превосходит первого члена ряда .\\

80. Если функция ${\displaystyle f(x)}$ и ${\displaystyle g(x)}$ дифференцируемы в окрестности точки ${\displaystyle x_0}$ и, кроме того, ${\displaystyle \lim _{x \to x_0} f(x)=0}$ и ${\displaystyle \lim _{x \to x_0} g(x)=0}$ причем ${\displaystyle g^{\prime}\neq 0}$ в окрестности точки ${\displaystyle x_0}$, то тогда ${\displaystyle \lim _{x \to x_0} \dfrac{f(x)}{g(x)}=\lim _{x \to x_0} \dfrac{f^{\prime}(x)}{g^{\prime}(x)}}$ при условии, что второй предел существует .\\

81. Пусть функции ${\displaystyle f(x)}$ и ${\displaystyle g(x)}$ определены и дифференцируемы в промежутке ${\displaystyle (a,b)}$, ${\displaystyle g^{\prime}(x) \neq 0}$ для всех ${\displaystyle x\in (a,b)}$, ${\displaystyle \lim \limits _{x \to a} f(x)=0}$ и ${\displaystyle \lim \limits _{x \to a} g(x)=0}$  существует конечный или бесконечный предел ${\displaystyle \lim \limits _{x \to a} \dfrac{f^{\prime}(x)}{g^{\prime}(x)}}$, тогда ${\displaystyle \lim \limits _{x \to a}\dfrac{f(x)}{g(x)}=\lim \limits _{x \to a} \dfrac{f^{\prime}(x)}{g^{\prime}(x)}}$ .\\

82. Если последовательность ${\displaystyle \left \{ x_{n} \right \}}$ такая, что для любого натурального значения ${\displaystyle n}$, ${\displaystyle y_{n} \leq x_{n} \leq z_{n}}$ и , то и ${\displaystyle \lim \limits _{n \to \infty} y_n=\lim \limits _{n \to \infty} z_n=A}$, то и ${\lim \limits _{n \to \infty} x_n=A}$ .\\

83. Если последовательность ${\displaystyle a_{n}}$ такая, что ${\displaystyle b_{n}\leqslant a_{n}\leqslant c_{n}}$ для всех ${\displaystyle n}$,причём последовательности ${\displaystyle b_{n}}$  и  ${\displaystyle c_{n}}$ имеют одинаковый предел при ${\displaystyle n\to \infty }$, то ${\displaystyle \lim _{n\to \infty }b_{n}=\lim _{n\to \infty }c_{n}=A\Rightarrow \lim _{n\to \infty }a_{n}=A }$ .\\

84. Если ${\displaystyle f(x)}$ непрерывна на промежутке ${\displaystyle [a,b]}$, то между точками ${\displaystyle a}$ и ${\displaystyle b}$ найдется хотя бы одна точка  ${\displaystyle \xi}$ такая, что будет иметь место равенство ${\displaystyle \int \limits _{a}^{b} f(x) dx=f(\xi)\cdot (b-a)}$ .\\

85. Пусть функция ${\displaystyle f(x)}$ непрерывна на отрезке ${\displaystyle [a,b]}$, тогда существует точка ${\displaystyle \xi \in [a,b]}$ такая, что ${\displaystyle \int \limits _{a}^{b} f(x) dx=f(\xi)\cdot (b-a)}$ .\\

86. Если неубывающая (невозрастающая) последовательность ${\displaystyle \{x_n\}}$ ограничена сверху (снизу), то она сходится .\\

87. Любая монотонная ограниченная последовательность ${\displaystyle \{x_n\}}$ имеет конечный предел, равный точной верхней границе, ${\displaystyle sup \{x_n\}}$ для неубывающей и точной нижней границе, ${\displaystyle inf \{x_n\}}$ для невозрастающей последовательности .\\

88. Пусть функция ${\displaystyle f(x)}$ абсолютно интегрируема на конечном или бесконечном интервале ${\displaystyle (a,b)}$, тогда ${\displaystyle \lim \limits _{\omega \to \infty} \int \limits _{a}^{b} f(x)\cdot \sin (\omega\cdot x)*dx=0}$ .\\

89. Если функция ${\displaystyle f}$ интегрируема на промежутке ${\displaystyle [a,b]}$, то ${\displaystyle \lim \limits _{p \to +\infty} \int \limits _{a}^{b} f(x)\cdot \sin (p\cdot x)*dx=0}$ .\\

90. Пусть последовательность функций сходится по мере к функции ${\displaystyle f}$ на ${\displaystyle E}$, тогда из неё можно выделить подпоследовательность, которая сходится почти всюду на ${\displaystyle E}$ к ${\displaystyle f}$.\\

91. Если последовательность функций ${\displaystyle f_n}$ сходится по мере к ${\displaystyle f}$, то у неё существует подпоследовательность ${\displaystyle f_{n_k}}$, сходящаяся к ${\displaystyle f}$ почти всюду .\\

92. Если функция ${\displaystyle f}$, определена на открытом множестве ${\displaystyle \Omega }$ евклидова пространства, ${\displaystyle A\subset \Omega }$ и ${\displaystyle \varlimsup _{x\to a}{\frac {|f(x)-f(a)|}{|x-a|}}<\infty }$ для всех ${\displaystyle a\in A}$, тогда ${\displaystyle f}$ дифференцируема почти везде в ${\displaystyle A}$ .\\

93. Пусть ${\displaystyle g}$, определена на открытом множестве ${\displaystyle \Omega }$ пространства ${\displaystyle \mathbb {R}^{n}}$, ${\displaystyle A\subset \Omega }$ и ${\displaystyle \varlimsup _{y\to a}{\frac {|g(y)-g(a)|}{|y-a|}}<\infty }$ для всех ${\displaystyle a\in A}$, тогда ${\displaystyle g}$ дифференцируема почти везде в ${\displaystyle A}$ .\\

94. Пусть ${\displaystyle a_{n}}$ и ${\displaystyle b_{n}}$  — две последовательности вещественных чисел, причём ${\displaystyle b_{n}}$ положительна, неограничена и строго возрастает, тогда, если существует предел ${\displaystyle \lim \limits _{n\to \infty }{\frac {a_{n}-a_{n-1}}{b_{n}-b_{n-1}}}}$, то существует и предел ${\displaystyle \lim \limits _{n\to \infty }{\frac {a_{n}}{b_{n}}}}$, причём эти пределы равны .\\

95. Если ${\displaystyle a_{n}}$ и ${\displaystyle b_{n}}$ последовательности действительных чисел, ${\displaystyle b_{n}}$ строго монотонна, неограничена и строго возрастает и существует предел ${\displaystyle \lim \limits _{n\to \infty }{\frac {a_{n}-a_{n-1}}{b_{n}-b_{n-1}}}=l}$, тогда ${\displaystyle \lim \limits _{n\to \infty } \frac {a_{n}}{b_{n}}=l}$ .\\

96. Если вещественная функция, непрерывная на отрезке ${\displaystyle [a,b]}$ и дифференцируемая на интервале ${\displaystyle (a,b)}$ , принимает на концах отрезка ${\displaystyle [a,b]}$ одинаковые значения, то на интервале ${\displaystyle (a,b)}$ найдётся хотя бы одна точка, в которой производная функции равна нулю.\\

97. Пусть функция  дифференцируема в открытом промежутке , на концах этого промежутка сохраняет непрерывность и принимает одинаковые значения: ${\displaystyle f(a)=f(b)}$ , тогда существует точка  , в которой производная функции  равна нулю : ${\displaystyle f'(c)=0}$  .\\

98. Пусть функция ${\displaystyle f(x)}$ непрерывна на отрезке ${\displaystyle [a,b]}$ и дифференцируема на интервале ${\displaystyle (a,b)}$, причем ${\displaystyle f(a)=f(b)}$ , тогда существует точка ${\displaystyle c \in [a, b]}$ такая, что ${\displaystyle f'(c)=0}$ .\\

99. Если функция f непрерывна на ${\displaystyle [a,b]}$, функция f дифференцируема во всех внутренних точках ${\displaystyle [a,b]}$ и ${\displaystyle f(a)=f(b)}$, тогда существует точка ${\displaystyle c\in (a,b)}$, в которой ${\displaystyle f^{\prime}(c)=0}$ .\\

100. Если функция ${\displaystyle f(x)}$  непрерывна на отрезке ${\displaystyle [a,b]}$, в каждой точке интервала ${\displaystyle (a,b)}$ существует конечная производная ${\displaystyle f^{\prime}(x)}$ и, кроме того,${\displaystyle f(a)=f(b)}$, то тогда между точками ${\displaystyle a}$ и ${\displaystyle b}$ найдется хотя бы одна точка ${\displaystyle с}$,${\displaystyle a<c<b}$ такая, что ${\displaystyle f^{\prime}(c)=0}$ .\\

101. Если функция ${\displaystyle f(x)}$ определена на сегменте ${\displaystyle [a,b]}$ и является монотонной на этом сегменте, то она может иметь на этом сегменте только точки разрыва первого рода, причем множество всех ее точек разрыва не более чем счетно .\\

102. Если функция ${\displaystyle f}$  определена на сегменте  ${\displaystyle \left [ a,b \right ]}$ и монотонна, то она может иметь внутри этого сегмента, точки разрыва первого рода, и число этих точек либо конечно, либо счётно .\\

103. Если функция ${\displaystyle f}$ интегрируема на ${\displaystyle [a,b]}$ и непрерывна в точке ${\displaystyle x_0 \in [a,b]}$, то функция ${\displaystyle F(x)=\int \limits _{a}^{x} f(t)\,dt}$  дифференцируема в точке функция ${\displaystyle x_0}$ и функция ${\displaystyle F^{\prime}(x_0)=f(x_0)}$ .\\

104. Если функция ${\displaystyle f(x)}$ непрерывна на промежутке ${\displaystyle [a,b]}$, то интеграл с переменным верхним пределом  ${\displaystyle \int \limits _{a}^{x} f(t)\,dt}$ имеет производную, равную значению подынтегральной функции при верхнем пределе, т.е ${\displaystyle \left (\int \limits _{a}^{x} f(t)\,dt\right )^{\prime}=f(x)}$ .\\

105. Если функция ${\displaystyle f}$ интегрируема на ${\displaystyle [a,b]}$ и непрерывна в точке ${\displaystyle с \in [a,b]}$, то интеграл с переменным верхним пределом ${\displaystyle F(x)=\int \limits _{a}^{x} f(t)\,dt}$ имеет производную в точке ${\displaystyle с}$ и выполняется равенство ${\displaystyle F^{\prime}(c)=f(c)}$.\\

106. Если ${\displaystyle \textstyle f(x)}$ непрерывна на отрезке ${\displaystyle \left [a,b\right ]}$  и ${\displaystyle \textstyle \Phi (x)}$ — любая её первообразная на этом отрезке, то имеет место равенство ${\displaystyle \int \limits _a^b f(x)\,dx=\Phi (b)-\Phi (a)={\Bigg .}\Phi (x){\Bigg |}_{a}^{b}}$ .\\

107. Если  ${\displaystyle F (x)}$ – любая первообразная функции ${\displaystyle f(x)}$,  то справедливо равенство ${\displaystyle \int \limits _a^x f(t)\,dt=F(x)-F(a)}$ .\\

108. Пусть ${\displaystyle f}$ - функция, интегрируема по Риману на отрезке ${\displaystyle [a,b]}$ и функция ${\displaystyle F}$ непрерывна на отрезке ${\displaystyle [a,b]}$ и дифференцируема в каждой внутренней точке этого отрезка, причем ${\displaystyle F^{\prime}=f(x)}$ , ${\displaystyle a<x<b}$, тогда справедлива формула ${\displaystyle \int \limits _{a}^{b} f(x)\,dx=F(b)-F(a)}$ .\\

109. Если функция ${\displaystyle y=f(x)}$ определенна на отрезке ${\displaystyle [a,b]}$ и на этом отрезке ${\displaystyle f(x)}$ дифференцируема ${\displaystyle n}$ раз, тогда ${\displaystyle f(x)}$ может быть представлена в виде ${\displaystyle f(x)=f(a)+\dfrac{f^{\prime}(a)}{1!}*(x-a)+\dfrac{f^{''}(a)}{2!}*(x-a)^2+\ldots+\dfrac{f^{(n-1)}(a)}{(n-1)!}*(x-a)^{n-1}+R_n}$ .\\

110. Пусть ${\displaystyle k \geq 1}$ является целым, и пусть функция ${\displaystyle f : \mathbb{R} \rightarrow \mathbb{R}}$ является ${\displaystyle k}$ раз дифференцируемой в точке ${\displaystyle a \in \mathbb{R}}$, тогда существует функция ${\displaystyle h_k: \mathbb{R} \rightarrow \mathbb{R}}$ такая, что ${\displaystyle f(x)=f(a)+\dfrac{f^{\prime}(a)}{1!}*(x-a)+\dfrac{f^{''}(a)}{2!}*(x-a)^2+\ldots+\dfrac{f^{(k)}(a)}{k!}*(x-a)^k+h_k*(x-a)^k}$, ${\displaystyle \lim \limits _{x \to a} h_k(a)=0}$ .\\

111. Пусть функция ${\displaystyle f(x)}$ ${\displaystyle n}$ раз дифференцируема в точке ${\displaystyle x_0}$, тогда ${\displaystyle f(x)=\sum \limits _{k=0}^{n} \dfrac{f^{(k)}(x_0)}{k!}*(x-x_0)^k+o((x-x_0)^n)}$ .\\

\end{document}